\documentclass{texMemo}
\usepackage{url}
\memoto{Drs. Gabauer and Midkiff}
\memodate{October 31, 2012}
\memofrom{BU ProPANE Team}
\memosubject{Project Status}
\begin{document}
\setlength{\parindent}{0pt}
\maketitle
\noindent
The purpose of this memo is to present the progress on the BU ProPANE project as well as summarize relevant information from the second panel. \\

The following list contains items that were to be completed in the past work period as well as the progress made by the BU ProPANE team. Please note that all of the items were completed.
\begin{itemize}
	\item Finish technical specifications document
		\subitem{Approved by Professor Thompson}
	\item Complete research portion of background document
		\subitem{Completed}
	\item Complete related technologies portion of background document
		\subitem{Completed}
	\item Perform trade analysis to determine optimal capture system 
		\subitem{Image comparison has begun (see slideshow)}
	\item Begin research on image processing services 
		\subitem{Investigated PIL and ImageJ (see slideshow)}
	\item Contact Microsoft in hopes of obtaining source code 
		\subitem{Email was sent}
\end{itemize}

The current state of the project is to configure the development environment that has been provided and tackle the image processing/Android development tasks. The goal for the remainder of the semester is to provide deliverables that each demonstrate that a single component of the system is feasible. Basically, the deliverables are ``proof of concept" tasks. For more information regarding the individual tasks, refer to the Gantt chart on the Deliverables page on the official BU ProPANE website (\url{https://sites.google.com/site/bupropane/deliverables}).\\ 
\\
The main topics of the panel (in addition to the progress and future plans) were: test procedures, image processing library selection, and post-analysis interface. 


\end{document}
