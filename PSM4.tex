\documentclass{texMemo}
\memoto{Drs. Gabauer and Midkiff}
\memodate{February 20, 2013}
\memofrom{BU ProPANE Team}
\memosubject{Project Status}
\begin{document}
\maketitle
\noindent
The purpose of this memo is to present the progress on the BU ProPANE project as well as summarize relevant information from the fourth panel. \\
\\
At the fourth panel, the ProPANE team demonstrated the three main components of the overall ProPANE system. The first demonstration showed the capture system app running on the Samsung Galaxy Camera. The app captured images every 5 seconds, separated capture events into timestamped folders, and allowed for zooming. The second demonstration showed how the end user interacts with the analysis system. The end user copies the images from the camera to their computer via USB then uploads them to a folder on a server to start the analysis system. The backend system detected the new set of images and started the third demo. The third demo showed the results of the analysis system and how it was classifying cells. \\
\\
The most significant result from the work performed since the last panel is that the analysis system can process an entire set of images from a 52 minute class in approximately half an hour. Additional results to note: the capture app is functioning correctly and the backend for the system is functioning correctly. \\
\\
The following are a list of items to be completed before the next panel:
\begin{itemize}
	\item Implement key image detection
	\item Implement key image composition
	\item Test prototype against technical specifications
\end{itemize}

\end{document}
